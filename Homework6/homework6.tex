\documentclass[UTF8]{ctexart}
\usepackage{amsmath}
\usepackage{amssymb}
\usepackage{amsthm}
\usepackage{indentfirst}
\usepackage{graphicx}
\title{作业6:决策树和继承学习}
\author{助教邮箱:wanghc17@mails.tsinghua.edu.cn}
\date{}
\usepackage{geometry}

\geometry{left=2cm,right=2cm,top=2cm,bottom=2cm}

\usepackage{color}
\linespread{1.5}



\begin{document}
\maketitle

\section{Bagging}

在课堂上我们已经了解到,Bagging方法可以减小模型的误差。我们现在从理论上来证明这一点。首先我们在数据集合$D$中有放回的抽样生成了$m$个数据集$\{D_{i}\}_{i=1}^{m}$,在每个数据集合$D_{i}$上训练分类器$h_{i}(x)$,假设我们现在有$n$个待预测的新样本$\{x_{j}\}_{j=1}^{n}$。对于其中的任意一个样本$x$,Bagging方法的预测值可以定义为多个分类器的预测平均值:
$$
h_{B} = \frac{1}{m}\sum_{i=1}^{m}h_{i}(x)
$$

若对于样本$x$真实的预测值为$y(x)$,则可以知道每个分类器$h_{i}(x)$的误差$\epsilon$为:
$$
\epsilon_{i}(x) = h_{i}(x) - y(x)
$$

对于m个单独的分类器,它们的平均均方误差可以定义为:
$$
E_{h} = \frac{1}{m}\sum_{i=1}^{m}\{\frac{1}{n}\sum_{j=1}^{n}[\epsilon_{i}(x_{j})]^{2}\}
$$

对于Bagging分类器的均方误差可以定义为:
$$
E_{h_{B}} = \frac{1}{n}\sum_{j=1}^{n}[\epsilon_{B}(x_{j})]^{2} = \frac{1}{n}\sum_{j=1}^{n}[h_{B}(x_{j}) - y(x_{j})]^{2}
$$

(1) 假设所有分类器的误差均值为零,而且互不相关,即:
$$
\frac{1}{n}\sum_{j=1}^{n}\epsilon_{i}(x_{j}) = 0.(i\in\{1,2,...,m\})
$$

$$
\frac{1}{n}\sum_{j=1}^{n}\epsilon_{i}(x_{j}) \epsilon_{k}(x_{j})= 0.(i,k\in\{1,2,...,m\})
$$

请证明:
$$
E_{h_{B}} = \frac{1}{m}E_{h}
$$

(2)但在实际情况中,往往它们的误差是高度相关的,请在(1)条件不满足的情况下证明:
$$
E_{h_{B}} \leq E_{h}
$$




\section{决策树}
实现决策树算法,并且在Sogou Corpus数据集上测试它的效果。

(注:附件中,$Sogou-webpage.mat$ 存储有wordMat和doclabel两个变量。前者为特征矩阵,大小为14400 * 1200,即包含14400个数据,每行数据包含1200维特征;后者为14400个数据的标签。可以使用predeal.py完成数据载入)

要求:

1. 请自己编写一种决策树算法。

2. 将数据\textbf{随机}分为3:1:1的三份,分别为训练集、交叉验证集、测试集。请在训练集上训练,交叉验证机上选择超参数,并在测试集上给出测试效果。因此,需在报告中给出超参数的选择,以及不同超参数下,训练集、交叉验证集的分类正确率,给出最好的超参数设置,并在测试集上给出测试效果。

3. 请在编写程序时,必须包含但不限于以下的几个函数:

(1) \textbf{GenerateTree(args)}:

\#生成树的总代码,args为各种超参数,请自由选择各类影响树性能的超参数。

(2)  \textbf{SplitNode(samlesUnderThisNode,thre,...)}:

\#对当前节点进行分支,\textbf{samlesUnderThisNode}是当前节点下的样本,\textbf{thre}是停止分支的阈值,停止分支的条件应在实验报告中说明。

(3)  \textbf{SelectFeature(samlesUnderThisNode,...)}:

\#对当前节点下的样本,选择待分特征。

(4)  \textbf{Impurity(samples)}:

\#给出样本\textbf{samples}的不纯度,请在实验报告中说明采用的不纯度度量。

(5)  \textbf{Decision(GeneratedTree, SamplesToBePredicted)}:

\#使用生成的树\textbf{GeneratedTree},对样本\textbf{SamplesToBePredicted}进行预测。

4. 请同学们尝试使用sklearn中的DecisionTreeClassifier与RandomForestClassifier函数,应用于该数据集中,将自己编写的决策树与这两种方法的测试集正确率进行对比,并做简要分析。

5. 有兴趣的同学,可以对树进行剪枝操作,实现一种剪枝方法,提升树的分类能力。(此项不做要求)





















\end{document}
